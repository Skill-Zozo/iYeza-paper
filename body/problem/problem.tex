\subsection{Implications}
\subsubsection{Financial}
Recurring illnesses result in extended treatment procedures which require additional provision of health care resources which are sometimes more costly\cite{SAGov16, Vern01}. This avoidable cost is incurred by both the facility and the patient. It is therefore especially problematic in resource limited settings which one would find in informal settlements such as townships in Cape Town, South Africa.

 \subsubsection{Health}

 Perhaps the most damaging consequence of non-compliance is its role in undermining the effect that the prescribed medication would have on the patient's wellbeing\cite{Vern01}. For instance, a patient that were to stop taking his/her medication may show the same symptoms after some time, suggesting that there might have been a resurgence of the same infection that she was given medical prescriptions for. This avoidable instance would mean that the patient's treatment process is delayed and furthur more the infection might have mutated  to become resistant to the initial prescription necesitating more complex (in some cases, more costly) prescription drugs. A real life scenario of this is with patients being treated for either tubercolosis or HIV. Without loss of generality, let us consider the case of TB. Patients diagnosed with TB are given prescription drugs to combat that infection but if the patient does not adhere, the infection or strain of TB can mutate to a more resistant strain called drug-resistant tubercolosis (DR-TB) or even worsen to multidrug-resistant tubercolosis (MR-TB) of which is a severe case and may be incurable \footnote{There exists strains of multidrug resistant tubercolosis that as of 20/09/2017 were deemed incurable}\cite{SAGov16}. Such chronic illnesses are especially problematic in countries such as South Africa, where the ministry of health released a report in 2016\cite{SAGov16}that states that the country has the third largest population of TB patients. The same report\cite{SAGov16} furthur estimates that by 2025 there could be a total of 12.3 million people suffering from chronic illnesses such as HIV and TB in that country.

\subsection{Causes}
Non-adherence to a medical professional's prescription can take on many forms. These could be (but not limited to) 1) a patient failing to follow medical professional's prescriptions (be it in terms of dosages, times of administering or prematurely stopping the medication), 2) delay in reporting illness and 3) missing appointments or follow up sessions\cite{Vern01}. Looking at the forms non-adherence can take can help in identify it's causes. Subsequently, researchers have suggested reasons to try and explain these causes. These suggestions include 1) increased complexity of drug prescriptions \cite{Fernerty12}, 2) health literacy of patients (combined with a general lack of confidence in medical professional)\cite{Kali99} and lastly high costs associated with accessing the medical treatment\cite{Vern01}. To add on to the last point, a study conducted by Verneire et al\cite{Vern01}, patients who did not come for follow up sessions cited transport costs as reason why they could not come as the health care facilities were often too far away from patients. Furthermore, expensive treatment artefacts or procedures were also cited as being reasons why some patients decided not to heed a doctor's prescription as they could not afford the treatment. To elaborate further on the second suggestion that researchers's made; researchers have found out that patients possesing a higher understanding of medicine were more likely to stop taking their medication often suspecting a misdiagnosis by the medical professional or a general lack of confidence in his/her abilities as a medical professional\cite{WilsonWolf}. For the first suggestion, complex medical prescriptions, will be the focus of this paper. Houts et al\cite{Houts06} have suggested that compliance to a medical prescription is due to 3 cognitive concepts, namely the ability to recall and understand a prescription and how much attention the patient placed on following the prescription. The solution suggested to the problem of adherence in this paper will focus on addressing these 3 aspects.
