As previously stated, non-adherence presents a serious challenge in South Africa. Here we propose a solution that addresses non-adherence by touching on 2 of the 3 cognitive concepts mentioned by Houts et al.
\subsection{Recall}
To improve the ease at which patients are able to remember their medical prescription, we follow the conclusion that Lester et al came to, in that short messages are more effective and easier to remember than long winded narrations. Secondly, through our system, we aim to make the prescription available to patients at anytime. This again is done to remind the patients what the prescription is, in the case the patient has forgotten, easing the cognitive load placed upon patients.
\subsection{Comprehension}
Currently, a number of patients in South Africa rely on labels written in English on their medication as their source of reference when they want to remind themselves of the medical prescription they have received from the medical professional. The use of English in these labels carries a heavy assumption in that the audience/patients are comfortable in that language, this is not always the case for residents of South Africa.Furthermore these labels have been criticized by medical researchers for being too complicated for people outside the medical fraternity to understand, citing the frequent use of medical jargon as the reason behind this\cite{Houts06}. To this end, Davis et al have been recommended that medical professionals should use vocabulary that is accessible to the patient because ultimately he/she will be the one reading the presctiption when taking in the medication.

To address all these shortfalls of the system, we suggest the use of an application that will disseminate these medical prescriptions as simple and concise narations in isiXhosa, a native language in South Africa. We hope, in this way, that patients can easily learn and fully understand the importance of following their given prescriptions. The use of audio to aid in learning presents some interesting promises especially when one considers the Mayer's cognitive theory of multimedia learning. In his writing, Mayer suggests that visual and verbal communication allows for a deeper understanding of the content being taught\cite{WilsonWolf}. Perhaps this can be explained by the fact that people have been consuming information through dialogues from child birth and so it is a learning technique that is not difficult to follow\cite{Patel10}.
