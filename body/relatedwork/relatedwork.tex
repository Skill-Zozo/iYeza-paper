\subsection{Interventions and Adherence}
There have been a number of publications that looked at adherence. Most publications have focused on providing interventions during the treatment process by sending SMSes as reminders to patients to take in their medication\cite{Strand10, Lester10, Pop11}. All 3 studies, \cite{Strand10, Lester10, Pop11}, have seen varying increase in adherence rates but it remains to be seen if these interventions are economically feasible in resource limited settings in the longterm (well after their pilot phase). Furthermore, we believe there could be an added benefit of providing prescriptions through the use of audio as it could help patients better understand the medication. Another intervention study by Forster et al \cite{Forster08} examined the use of electronic medical record (EMR) systems in identifying HIV patients that are more likely to be non compliant. Patients who missed follow up appointments were identified in a timely manner so as to initiate the intervention strategies to boost adherence. The study does not have a conclusive answer as to whether or not the use of EMR systems does indeed increase adherence. This is partly due to the fact the EMR systems are not the actual intervention strategy but aid other strategies. The only conclusion that the researchers could draw was that more missing data often lead to more patients failing to attend appointments.

\subsection{Prescriptions and Adherence}
Verneire et al \cite{Vern01} have emphasized the need for medical professionals to help patients understand the prescription and as such have advocated for a more in depth analysis of the cognitive factors that play a role in a patient's adherence to the prescription. In \cite{Ostini14} researchers point out that a non-adherent patient that does not possess a high health literacy is more likely to cite forgetfulness rather than citing a lack of confidence in the medication or the medical professional as the reason why he or she took an incorrect dosage of the medication, in both cases supporting Verneire et al's earlier statements. In this vain, Davies et al \cite{Davis06} have looked at simplifying prescriptions labels through the use of pictorial aids and simpler language, they concluded that there are benefits to this as patients showed a clearer understanding of the contents of the prescription.

\subsubsection{Audio and Learning}
Audio systems have been the preferred medium of communication especially in developing communities. A successful example of this is the use of relayed audio messages in the Avaaj Otalo paper published by Patel et al \cite{Patel10}, where by the researchers setup a system that connected farmers in a rural region of India. The farmers were encouraged to pose questions and listen to responses from other farmers or listening to other professionals who could be giving out farming advice on the platform. The system showed signs of success in its phase even being used to share other community news. Although the researchers did not evaluate how much was learnt through the system, a number of users reported to have gained some form of education through the system be it about farming techniques or how to control pests. The same is true for other IVR services, like SpokenWeb \cite{Spoken10}, dispensing educational material through the use of audio, there is little emphasis placed on evaluating the learning impact of providing such systems to people.
An interesting study by Ramkrishnan et al\cite{DocTalk} has also decided to give patients access their medical prescriptions through voice recordings recorded by the patient's doctor. Unfortunately, the adoption of this system was poor with the authors citing the doctor's busy schedule as the main culprit. As such, they could not conduct experiments to report on potential benefits to audio systems in the medical fraternity. However, Joshi et al \cite{Joshi13} designed an intervention strategy, that makes use of an audio system to give out 'treatment advice' to people living with HIV and AIDS, and have since reported it to have a 'desired effect' on adherence rates. The system makes us of voice calls to patients to remind them about their medication, it also provides the patients with a voice driven interface where users can report any further symptoms that they could be displaying, or listen to upcoming appointments or health tips. This system would work well in the South African context, had it not been for the cost implications associated with voice calls and the fact that it relies heavily on good internet access at hospitals to work.

To this end, we have designed a system, similar to both the works of Joshi et al \cite{Joshi13} and Ramkrishnan et al\cite{DocTalk}. Similar to TAMA (TREATMENT ADVICE BY MOBILE ALERTS by Joshi et al) in that it also gives out medical prescriptions and monitors usage. And also, similar to DocTalk by Ramkrishan et al in that it uses voice recordings that can be played locally on the phone.
