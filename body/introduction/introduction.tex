Most patients consult medical pratictioners when they fall sick with the aim of receiving treatment for that illness. Typically, on the first day of the treatment process, the doctor would analyze the symptoms that the patient's body is displaying, provide a diagnosis and finally prescribe the medication that is suitable to combat the patient's illness. For the rest of the treatment process (which could stretch to 4 weeks or more), it becomes the patient's responsiblity to heed the medical practioner's advice and follow the prescription suggested. Research has shown that not all patients follow their given prescription to the letter, some even experiencing recurring infections of the same disease they were treated initially treated for\cite{Vern01}. This reluctance to follow a medical prescription has been coined as non-adherence, implying that adherence is the ease at which patients follow a medical prescription\cite{Vern01}. There have been several studies published on medical adherence, the first of which dating back to 1968\cite{Fernerty12}. There are two common ways in which researchers have defined medical compliance. Frequently adherence is given the previously mentioned definition in that it is seen as the degree to which a patient follows a medical professional's drug prescriptions\cite{Vern01, Fernerty12}. Others choose to view compliance from a result perspective, whereby medical professionals measure a patient's compliance by evaluating the state of the patient's health. Another definition looks at the extent at which a patient's lifestyle changes in order to comply with a medical professional's prescription \cite{Vern01}. All of these definitions are primarily concerned about getting insights as to what happens after the doctor-patient consultation has been concluded, all the way through to the end of the medication treatment process.

Patients that incorrectly administer their medication are likely to fall sick again to the same illness and requiring additional medical resources for treatment\cite{Vern01, Kali99}. Avoidable instances such as these seek to undermine the level of healthcare that can be offered by the facility as the facility would undoubtedly incur costs in providing the same patient with more medication, medication that could have been efficiently dispensed to other patients that need them\cite{Vern01}. As a result, there has been a growing interest within the medical research fraternity to try and understand what would cause patients to not follow the prescriptions given by their medical practitioner. Researchers have looked into ways of improving doctor-patient consultations with the hope that patients have an increased appreciation of the importance of satisfying a medical practitioner's medical prescription.

The authors of this paper have chosen to focus on the medical prescriptions given out by doctors with furthur emphasis on the understanding the format these prescriptions could take when addressing patients who are not first language English speakers. We set out to acertain if there are any benefits to giving patients their medical prescriptions, through audio, in a language they are comfortable in, in this case isiXhosa. To evaluate if giving patients prescriptions in a local language would be beneficial, we sought out to test a patient's understanding of the prescription and how easily they could remember the contents of the prescription after 5 days. We also used an app to simulate and monitor a patient administering medication during the same period, with the aim of reporting on any benefits or setbacks on adherence. This paper has five sections, firstly its the introduction followed a furthur analysis presented by non-adherenceby previous research into adherence, followed by section 3 that details how the study was conducted then comes the results and discussion and finally the conclusion.
