\documentclass[sigconf]{acmart}

\usepackage{booktabs} % For formal tables


% Copyright
%\setcopyright{none}
\setcopyright{acmcopyright}
%\setcopyright{acmlicensed}
% \setcopyright{rightsretained}
%\setcopyright{usgov}
%\setcopyright{usgovmixed}
%\setcopyright{cagov}
%\setcopyright{cagovmixed}


% DOI
\acmDOI{10.475/123_4}

% ISBN
\acmISBN{123-4567-24-567/08/06}

%Conference
\acmConference[WOODSTOCK'97]{ACM Woodstock conference}{July 1997}{El
  Paso, Texas USA}
\acmYear{1997}
\copyrightyear{2016}


\acmArticle{4}
\acmPrice{15.00}

% These commands are optional
%\acmBooktitle{Transactions of the ACM Woodstock conference}
\editor{Banele Matsebula}

\begin{document}
\title{The Effect of Audio Usage in Administering Drug Prescriptions to Low Literate Patients on Medical Adherence}

\author{Banele Matsebula}
\affiliation{%
  \institution{University of Cape Town}
}
\email{mtsban004@myuct.ac.za}

\author{Thuto Aphiri}
\affiliation{%
  \institution{University of Cape Town}
}
\email{aphthu001@myuct.ac.za}

\author{Lerato Mosegedi}
\affiliation{%
  \institution{University of Cape Town}
}
\email{msgler002@myuct.ac.za}

\begin{abstract}
This is a review on existing research on medical adherence with an added emphasis on understanding the causes and effects of non adherence in low literate patients. To address the issue of non adherence, researchers have conducted studies investigating various approaches, some of which have looked into improving drug prescriptions that patients receive. In this review we look at the challenges that are unique to low literate patients when it comes to understanding their drug prescriptions and we compare the benefits and drawbacks of using either audio, text or pictures when delivering the prescription. We conclude that having patients call to listen to audio messages about their drug prescriptions would be the optimum solution as audio is more understandable (to low literate patients) compared to text and it is more feasible to implement when compared to using moving pictures or static pictures.
\end{abstract}

%
% The code below should be generated by the tool at
% http://dl.acm.org/ccs.cfm
% Please copy and paste the code instead of the example below.
%
\begin{CCSXML}
<ccs2012>
 <concept>
  <concept_id>10010520.10010553.10010562</concept_id>
  <concept_desc>Computer systems organization~Embedded systems</concept_desc>
  <concept_significance>500</concept_significance>
 </concept>
 <concept>
  <concept_id>10010520.10010575.10010755</concept_id>
  <concept_desc>Computer systems organization~Redundancy</concept_desc>
  <concept_significance>300</concept_significance>
 </concept>
 <concept>
  <concept_id>10010520.10010553.10010554</concept_id>
  <concept_desc>Computer systems organization~Robotics</concept_desc>
  <concept_significance>100</concept_significance>
 </concept>
 <concept>
  <concept_id>10003033.10003083.10003095</concept_id>
  <concept_desc>Networks~Network reliability</concept_desc>
  <concept_significance>100</concept_significance>
 </concept>
</ccs2012>
\end{CCSXML}

\ccsdesc[500]{Human-centered computing}
\ccsdesc[300]{Accessibility}
\ccsdesc{Accessibility design}
\ccsdesc[100]{Evaluation methods}

\keywords{Audio Messages, Medical Adherence, Medical Non-compliance, Low Literate, Interventions}

\maketitle

\input{samplebody-conf}

\bibliographystyle{ACM-Reference-Format}
\bibliography{sample-bibliography}

\end{document}
